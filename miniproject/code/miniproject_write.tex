\documentclass[11pt]{article}
\title {Quantifying the overlap of sharks, enforcement and illegal fishing inside one of the world’s largest marine protected areas}
\author{Emma Deeks}
\date{10 Dec 2019}
\usepackage[margin=2cm]{geometry}
\usepackage{graphicx}
\usepackage[]{lineno}
\usepackage[backend=biber,style=authoryear,bibencoding=utf8]{biblatex}
\usepackage{setspace}
\setlength{\parindent}{0em}

% bibliography secion %

\addbibresource{miniproject.bib}

\begin{document}
	
	\begin{titlepage}
		
		
		\centering % this centers everything on the page
		
		%\vspace  % Whitespace at the top of the page 
		
		
		
		% --------------------------
		%% TITLE
		
		\vspace*{5\baselineskip}
		
		\rule{\textwidth}{1.6pt}\vspace*{-\baselineskip}\vspace*{2pt} % Thick horizontal rule
		\rule{\textwidth}{0.4pt} % Thin horizontal rule
		
		\vspace{0.75\baselineskip} % Whitespace above the title
		
		{\LARGE How well do different mathematical models, based upon foraging theory (mechanistic) principles vs. phenomenological ones, fit to functional responses data across species? \\} 
		
		\vspace{0.75\baselineskip} % Whitespace below the title
		
		\rule{\textwidth}{0.4pt}\vspace*{-\baselineskip}\vspace{3.2pt} 
		\rule{\textwidth}{1.6pt} 
		
		\vspace{2\baselineskip} 
		
		% ---------------------------------
%% SUPERVISORS & CONTACT EMAIL

Student: \\
Emma Alice Deeks \\ 
	
		\vspace{1.5 \baselineskip} % Whitespace between text
		
		Contact: \\
		ead19@imperial.ac.uk
		

	\end{titlepage}
	
	\linenumbers
	\doublespacing
	\section{Introduction}
	\noindent

	
Functional responses describe the way in which a predator responds to changing densities of prey \cite{Holling1959}. This is important as it is a key regulating factor for population dynamics. The rate at which a predator kills its prey at different densities ultimately determines the efficiency of a predator in regulating population densities of prey. \cite{Murdoch1975}. \cite{Holling1959} described three types of functional responses curves, the first, an increasing linear relationship indicating a constant rate of predation. The second is a decelerating curve which indicates saturation at a certain level, e.g. saturation of predator eating prey, and the third is a sigmoidal relationship which indicates an increased rate of prey killing, or, positive density dependant prey mortality. \cite{Pervez2005}. It has been observed that often, the type II functional response is favoured in predator prey relationships, that is a curve which starts at low prey densities and as the prey density increases so does the predation rate linearly until it saturates at an upper limit \cite{Jeschke2002}. Functional response models are used, often above linear models, as they incorporate two parameters which are handling time (time required to attack, consume and digest prey) and attack/search efficiency of a predator to a prey \cite{Fathipour2016}. Functional response models can be applied in ecology when predicting future population densities of predators and prey within the functional response relationship as well as what happens to each member of the relationship when densities of one changes \cite{Jeschke2002}

Functional responses can be modelled in a phenomenological way, such as the Holling type II model \cite{Holling1959}, which cannot explain the underlying mechanism of the curve they produce, the parameters cannot be mechanistically explained, but still produce the traditional type II response curve. The second model type is the mechanistic model where the parameters involved can all be mechanistically explained have are developed to describe a biological mechanism \cite{Jeschke2002}.

Whilst these models are widely used to fit functional responses between species, many other factors can potentially effect the functional responses between species in space and time. The shape of the functional response curve is dependent on several characteristics such as the encounter rare, capture efficiency and the handling time of the predator \cite{Holling1965}, these various characteristics are often species dependant as species vary in a number of attributes such as size and space \cite{Elliott2003}. To account for these varying characteristics studies such as \cite{Aljetlawi2004}  have included body size into functional responses to produce a size dependant functional response. Other studies such as \cite{Pawar2012} have highlighted the importance of considering the dimensions in which consumers encounter resources, for instance, 2D dimensions involve terrestrial habitats whereas 3D dimensions include aerial and pelagic encounters. Studies such as these highlights the importance of not only choosing the best model to accurately describe functional response, but to consider the importance of species on a more individual basis, for instance, the habitat they occupy and the dimension in which predators encounter resources. 

This study looks at two mechanistic and two phenomenological models and fits them to a functional response dataset from \cite{Pawar2012} study in order to find whether mechanistic or phenomenological best describes functional response data. This study also looks at the dimensions and habitats the species included in the functional response curves are in and aims to test whether the model type varies based on this.

The key questions of this report are:
	\begin{itemize}
	\item How well do different mechanistic and phenomenological mathematical models fit functional response data across different species
	\item How does habitat effect these model fits? 
	\item How does the dimension the predator encounters the resource effect model fits? 
	\end{itemize}
\newpage

	\section{Methodology}
\noindent

Using \cite{Pawar2012} functional response dataset, four models were fitted to 214 functional response curves. The models included the cubic, quadratic, Holling Type II and Generalised functional response models (Fig.1). The consumption rate was plotted against the resource density and the resulting curve was fitted. AIC and BIC values were used to determine the best fit model as the lowest value was taken.
The starting values for the Holling Type II and the Generalised Functional Response model were calculated for each individual curve as well as optimised for each individual curve. When optimising the starting value for handling time, both the mean and median value was calculated and whichever gave the lowest AIC value was used as the optimised starting value. The slope of the curve from the data points below the h value gave the search rate, (a). The General Functional Response model had an additional spatial parameter (q) which was kept at 0.78 as used by \cite{Pawar2012}. In this study the phenomenological quadratic model was viewed as the phenomenological Type II functional response and the cubic, Type III. 
To incorporate an ecological component to the study, the data was subset by habitat and resource density to study questions 2 and 3 respectively. \newline

\begin{figure}[h!]
	\centering 
	\includegraphics{../results/Example_modelplot.pdf}
	\caption{Illustration of four models}
	\label{Four models plotted}
\end{figure}



Computational tools: \newline



The first script in the workflow is a python script that;
	\begin{itemize}
	\item Imports data and subsets based on relevant columns for study, removes any missing values and saves the resulting dataframe to a csv file 
	\end{itemize}

The second script in the workflow is an R script that;
\begin{itemize}
	\item Imports the modified csv file and nests the data using the 'dplyr' package 
	\item It then runs a function for obtaining initial start values on the nested data and outputs a dataframe of start values called 'datatouse'
	\item It defines the functions of the Holling Type II functional response as 'powMod' and the generalised functional response model as 'GenMod'
	\item This script then optimises starting values to input into a for loop to calculate the best model fit for each ID 
	\item The outputs of this loop are two tables, 'optimisedtable.csv', which has the best fit model for each of the functional response curves as well as the habitat and dimension the species is in. 'MergedOptTable.csv' has every model and their associated AIC values for statistics in the plotting script 
	\item Finally, this script plots all four models on a sample functional response model as an illustration of model fits. (Fig. 1)
\end{itemize}

The third script in the workflow is a python script that;
\begin{itemize}
	\item Imports 'MergedOptTable.csv' and 'optimisedtable.csv' for statistical analysis and plotting of data for results 
\end{itemize}


	\section{Results}
\noindent

Out of the 214 functional response curves, 134 of the functional response curves had the lowest AIC values of phenomenological model fits, a 20\% increase compared to mechanistic. The difference in model fits between mechanistic and phenomenological were significant ($\chi$\textsubscript{2}(DF = 2, N = 214) = 13.6, P $>$0.01).  This results indicates that most of the functional response curves fit the phenomenological models best, that being the quadratic and cubic models. (Fig.2) \newline

\begin{figure}[h!]
	
	\includegraphics[]{../results/mecphe.pdf}
	\caption{Resource Dimensionality Boxplot}
	\label{Model Type comparison}
	
\end{figure}

To infer whether habitat had a significant effect on the minimum AIC value a repeated measures ANOVA was conducted with a within subject factor of model and a between subject factor of habitat. There was a significant main effect of model (Two-way ANOVA: f\textsubscript{3, 20.437}, P = $>$0.01) and a significant main effect of habitat (Two-way ANOVA: f\textsubscript{2, 10.560}, P $>$0.01) on the minimumm AIC values (Fig.3). The interaction between the model used and habitat on the resulting minimum AIC value was not significant (Two-way ANOVA: f\textsubscript{6, 2.249}, P =0.022).  (Fig. 3). \newline A Tukey Post-Hoc analysis revealed all three habitats to be significantly different from one another. 

\begin{figure}[h!]
	\centering 
	\includegraphics{../results/HabitatBoxplot.pdf}
	\caption{Resource Dimensionality Boxplot}
	\label{Habitat Boxplot of model types}
\end{figure}

Similarly, to infer whether the dimension the resource was encountered in was significant in determining the minimum AIC value and thus, best model fit for each curve, a repeated measures ANOVA was conducted with a within subject factor of model and a between subject factor of resource dimension. There was a significant main effect of model (Two-way ANOVA: f\textsubscript{3, 19.912}, P = $>$0.01) and a significant main effect of the dimension the resource was encountered in (Two-way ANOVA: f\textsubscript{2, 22.788}, P $>$0.01) on the minimum AIC values (Fig.4). The interaction between the model used and the dimension of the resource on the resulting minimum AIC value was also not significant (Two-way ANOVA: f\textsubscript{6, 0.522}, P =0.017).  (Fig. 4). \newline A Tukey Post-Hoc analysis revealed all three dimensions, 2D, 3D and sessile,  to be significantly different from one another. 


\begin{figure}[h!]
	\centering 
	\includegraphics{../results/Res_Dim_Boxplot.pdf}
	\caption{Resource Dimensionality Boxplot}
	\label{Resource Dimensionality Boxplot of model types}
\end{figure}
\newpage


	\section{Discussion}
\noindent

Overall, the phenomenological models, quadratic and cubic models, fit 20\% more functional responses than mechanistic, the Holling Type II and the Generalised Functional response model. However, concluding that phenomenological models are the 'best fit' models for functional response curves is arguably a very general and, inaccurate, conclusion. Bolker (2008) stated that 'mechanistic models are more powerful since they tell you about the underlying processes driving patterns, they are more likely to work correctly when extrapolating beyond the observed conditions'. This statement holds a lot of significance since, mechanistic models are often used above phenomenological models due to their ability to model based on biological processes, the parameters in mechanistic models have been created in an attempt to describe a biological process, such as handling time or search rate used in the Holling Type II model. Phenomenological models are used to fit data, there is no biological relevance behind this fitting of data.  (HILBORN AND MANGEL 1997). Models that have biological interpretation have fundamentally more value (REFERENCEJohnson Omland 2004) \newline
Whilst this may be a valid explanation on why phenomenological models should not be immediately concluded as the model type that best fit functional response datasets, it does not offer a reason as to why these models fit best despite not having these biological parameters used by mechanistic models. A likely reason 60\% of functional response curves fit phenomenological models is because these models are simple. Simple models tend to have less bias, principles such as parsimony, which is the ratio of degrees of freedom between the alternative and null hypothesis (REFERENCES Marsh and Hau, 1998). When comparing models, parsimony favour the more 'simplistic' model, a model with fewer parameters and thus, more degrees of freedom (REFERENCE Mulaik, 1998). As a result of this, principles such as the parsimony principle penalise complex models such as mechanistic models that have parameters (REFERENCE Raykov and Marcoulides, 1999). Therefore, if the Holling Type II model and the phenomenological quadratic model had a similar fit to each other, which is highly likely as both models favour Type II functional response curves, the parsimony principle would opt for the phenomenological quadratic model. This may lend reason as to why these more simplistic models were favoured over the mechanistic models. That being said, both phenomenological and mechanistic models are attempting to model functional response curves, despite the debate on model fits both methods are useful in predicting how prey and predator populations are going to respond on density fluctuations and thus help inform management priorities. 
During the analysis stage of this project AIC and BIC values were calculated in order to best fit the models, upon inspection AIC and BIC outputs did not differ from one another and ultimately only AIC values were taken forward to be reported. However, future studies should consider appropriate measures of model fit. A study conducted by Aho, Derryberry and Peterson, 2014 found that BIC had a higher probability of selecting a true model compared to AIC upon higher sample sizes. 



Another aspect of the study which was overlooked was the taxa that the species were placed, the phylogeny of a species has been shown to impact types of functional responses, a study by Robert P.Dunn in 2020 discussed the importance of including taxa, especially of the predator, when looking at model fits of functional responses



Out of the 215 curves fitted with the four different models 60\% of the best fits were phenomenological models, in this study, the cubic and the quadratic model. Whilst it may be expected that the mechanistic models would be a more suitable fit due to the optimised start parameters, the handling time (h) and the            (a), as well as the spatial parameter encompassed in the generalised functional response model. Potentially, the method for optimising starting values of these models were involved in why these models were less successful then the mechanistic models. As well as this, the spatial parameter q was set to 0.75, as used in \cite{Pawar2012} paper of functional responses. Varying or optimising this spatial parameter may have potentially led to decreased model fits. As well as this, having the spatial parameter set to 0.75 means the line produced closely resembles a Type II functional resopnse, it may be that more models fitted the Type III functional response model, which is likely the case as the cubic model fitted           of the models overall. This would mean varying spatial parameters, and also looking at the type fo species within the functional response curves.    PROBLEMS WITH TYPE III FNCTIONAL RESPONSES INCLUDE 

	
\newpage


\vspace*{1\baselineskip}
\printbibliography 


\newpage

\end{document}